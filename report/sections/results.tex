Algorithm ended up beeing very successful. Robot is able to walk
smoothly on a flat terrain. What was more suprising is that robot is also able to
walk on an uneven terrain with small obstacles. The PID balancing controller
helped a lot with maintaining stability during walking.

Another practical implementation could be robots walking in a theme parks
by tweeking the gait parameters it is easy to achieve funny walking styles.
Overall, the spline-based trajectory generation approach proved to be
effective for controlling the quadruped robot's walking gait in simulation.

Robots like that arleady exists in a Disney theme park, from available information
their approach uses reinforcement learning to train the walking gait.
Figure~\ref{fig:disney-bipedal} shows a bipedal robot. Same approach is implemented
by Jan Górski.

\begin{figure}[htbp]
    \centering
    \includegraphics[width=0.45\textwidth]{imgs/disney_bipedal.png}
    \caption{Disney's Imagineering bipedal robot. \cite{disney-control-bipedal}}
    \label{fig:disney-bipedal}
\end{figure}



\subsection{Comparision to between Splines and RL algorithms}
Both spline-based trajectory generation and reinforcement learning (RL) approaches
have their own advantages and disadvantages when it comes to controlling quadruped
robot locomotion. Key difference is that spline-based approach is simpler to implement
and requires computing power of medium size calculator while RL approach requires
a lot of computing power to train the model and then run model in real-time. Despite 
this difference spline approach can be expanded and achieve similar results to RL approach.

\subsection{Reinforcement Learning [Jan Górski]}
This method involves training a neural network to learn optimal walking gaits
through trial and error. The robot receives rewards for successful movements and
penalties for failures, allowing it to gradually improve its performance over time.
To train the RL model Jan used SDU UCloud computing resources. Even with dedicated
GPU it took several hours to train the model. Researches from ETH Zürich 
\cite{deep-tracking-control} have demonstrated that RL approach can achieve
impressive results in quadruped locomotion, including walking on very challenging terrain. 
A video demonstrating their results can be found at \cite{deep-tracking-control-video}. 


\begin{figure}[htbp]
    \centering
    \includegraphics[width=0.45\textwidth]{imgs/Jan_RL_scheme.png}
    \caption{The motion imitation pipeline graphic from Jan Górski's report.}
    \label{fig:RL-scheme}
\end{figure}


RL approach has the advantage of being able to adapt to a wide range of terrains
and conditions, as the model can learn to generalize from its training data. 
What RL enables for researches is to prepare training data using motion capture
from animals. This way robot can imitate real animal movements.

\subsection{Spline foot trajectory}
My approach is based on generating smooth foot trajectories using cubic splines.
This method does not require high computing power nor require training. However,
witch a proper tuning and additional modules witch modifies the foot trajectory 
in real-time it can achieve similar results to RL approach. Researchers from Brazil
\cite{pedro2024quadruped} have demonstrated that spline-based trajectory generation
can be effective for controlling quadruped robots, including walking on uneven terrain.
Great advatange of spline approach is that it is predictable and stable. It is 
just a math function that could be analyzed and prooved to be stable. 

\subsection{Comparison with RL method}
As a good metric to compare both methods we used IMU data, to measure robot 
stability during walking on flat terrain. To do so we recorded Gyroscope
and Accelerometer data from both robots during walking. Figures~\ref{fig:Spline-robot-IMU-data}
and ~\ref{fig:RL-robot-IMU-data} shows the recorded data. It is visible
that in both cases robots achieves similar angular velocities but 
the accelerometer data shows that spline-based robot has lower
acceleration spikes. This suggests that the spline-based robot
is more stable during walking, which is a desirable trait for quadruped locomotion.

\begin{figure}[htbp]
    \centering
    \includegraphics[width=0.45\textwidth]{imgs/Gyro_Data.png}
    \includegraphics[width=0.45\textwidth]{imgs/ACC_data.png}
    \caption{IMU data from Spline Algorithm. Top: Gyroscope data, Bottom: Accelerometer data.}
    \label{fig:Spline-robot-IMU-data}
\end{figure}

\begin{figure}[htbp]
    \centering
    \includegraphics[width=0.45\textwidth]{imgs/Jan_Gyro.png}
    \includegraphics[width=0.45\textwidth]{imgs/Jan_ACC.png}
    \caption{IMU data from RL Algorithm. Top: Gyroscope data, Bottom: Accelerometer data.}
    \label{fig:RL-robot-IMU-data}
\end{figure}

Another important test we pergormed was a Fourier Transform analysis of the
robot's movement during walking. This analysis helps to identify the dominant
frequencies in the robot's motion, which can provide insights into its stability
and responsiveness. Figures~\ref{fig:Spline-robot-IMU-FFT-data} and ~\ref{fig:RL-robot-IMU-FFT-data}
shows the FFT results for both robots. In this case Spline controller shows 
amplitude spikes at higher frequencies where RL controller has most of the energy
concentrated at lower frequencies. This suggests that the RL-controlled robot
has a smoother and more stable gait, as lower frequency movements are generally
associated with better stability. That could be probably solved by tweeking PID
parameters of the balancing controller and PD controller for legs. In real life
application that would be very important as high frequency vibrations could
damage the robot's hardware over time. 


\begin{figure}[htbp]
    \centering
    \includegraphics[width=0.45\textwidth]{imgs/Gyro_FFT.png}
    \includegraphics[width=0.45\textwidth]{imgs/ACC_FFT.png}
    \caption{IMU Fourier analysis data from Spline Algorithm. 
    Top: Gyroscope data, Bottom: Accelerometer data.}
    \label{fig:Spline-robot-IMU-FFT-data}
\end{figure}

\begin{figure}[htbp]
    \centering
    \includegraphics[width=0.45\textwidth]{imgs/Jan_GYRO_FFT.png}
    \includegraphics[width=0.45\textwidth]{imgs/Jan_ACC_FFT.png}
    \caption{IMU Fourier analysis data from RL Algorithm. 
    Top: Gyroscope data, Bottom: Accelerometer data.}
    \label{fig:RL-robot-IMU-FFT-data}
\end{figure}