Quadruped robots are very complex to make. They are complex in a mechanical
design as well as in control algorithms. The goal of this project is to
implement walking algorithms for a quadruped robot. My focus will be on
generating smooth trajectories for the robot's legs using splines.
Jan Górski will be implementing the Trajectory Generation using reinforcement
learning approach. In the end, we will compare both methods and evaluate.
Both algorithm will be tested in a Unitree Mujoco simulation environment with
a Unitree GO2 robot model. Both algorithm should be able to walk forward on 
a flat and obstacle terrain.


\begin{figure}[htbp]
    \centering
    \includegraphics[width=0.45\textwidth]{imgs/mujacoo_sim.png}
    \caption{Unitree GO2 quadruped robot in the Mujoco simulation.}
    \label{fig:quadruped-go2}
\end{figure}


\subsection{Mujoco simulation enviroment}
Mujoco is a physics engine that is widely used for simulating robots maintained by 
DeepMind. Initial paper describing Mujoco \cite{mujoco} was published in 2012. 
It is known for its speed and accuracy in simulating complex robotic systems.

Unitree Robotics provides a Mujoco simulation environment for their GO2 robot
\cite{unitree-mujoco-github}. This environment includes a detailed model of the
robot, as well as various terrains and obstacles for testing walking algorithms.
Unitree provides exaple of Sim to Real transfer. That makes this project "testable"
in real world after the simulation phase.


\begin{figure}[htbp]
    \centering
    \includegraphics[width=0.45\textwidth]{imgs/robot_in_terrain.png}
    \caption{Unitree GO2 robot and rough terrain in Mujoco simulation.}
    \label{fig:mujacoo-sim-terrain}
\end{figure}


\subsection{Unitree GO2 Robot}
The Unitree GO2 robot \cite{unitree-go2} is a quadruped robot designed for research
and development in robotics. It features a lightweight design, high mobility, and
advanced sensors for navigation and obstacle avoidance. The robot is equipped with
12 degrees of freedom, allowing for complex movements and maneuvers. 
This robot is widely used in research due to its affordability and versatility. 
Cost for the robot for 1st of December 2025 is 1600 USD.

Unitree GO2 robot weighs 15kg and can output a maximum of 3kW of power. Each 
motor in a leg can provide up to 45Nm of torque. Each version of robot is eqquipped
in 8 core CPU and educational version could be equipped in Nvida Jetson Orin SBC. 
These parameters make the GO2 robot very capable platform for research in
quadruped locomotion. In my approach to control the robot high computing power
is not required as the trajectory generation could be precomputed and stored
in a look-up table.