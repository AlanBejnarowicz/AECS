\subsection{Gait phase generator}
The gait phase generator is responsible for coordinating the movements of the 
robot's legs during walking. It defines the possition of each leg in the gait cycle,
from given spline trajectory. The gait phase generator uses a time-based approach to determine
the phase of each leg, ensuring that the legs move in a coordinated manner.
Algorithm uses the following equation to calculate the phase position of each leg
\ref{phase-equation}:

\begin{equation}
\text{cycle\_pos} = \left( \frac{t}{T_{\text{cycle}}} + \text{offset} \right) \bmod 1
\label{phase-equation}
\end{equation}
Where \( t \) is the current time, \( T_{\text{cycle}} \) is the duration of a full gait cycle,
and \( \text{offset} \) is a phase offset for each leg to ensure proper

Current offset sequencing is in Table \ref{offset-table}. This sequencing is used by 
multipla animals such as dogs or horses while trotting. 

\begin{table}[h]
\centering
\caption{Gait phase offsets for each leg.}
\label{offset-table}
\begin{tabular}{c c}
\hline
\textbf{Leg} & \textbf{Phase Offset} \\
\hline
Front Right & 0.0 \\
Rear Left & 0.0 \\
Front Left & 0.5 \\
Rear Right & 0.5 \\
\hline
\end{tabular}
\end{table}




\subsection{Spline Trajectory Generation}
The spline trajectory generation module is responsible for creating actual foot 
trajectories. It uses cubic splines to interpolate between key waypoints, ensuring smooth
and continuous motion. The spline generation process involves defining control points
for each leg's trajectory and then using these points to create cubic spline functions.

\begin{figure}[htbp]
    \centering
    \includegraphics[width=0.45\textwidth]{imgs/leg_swing.png}
    \caption{Cubic spline trajectory for a leg swing motion.}
    \label{fig:spline-trajectory}
\end{figure}

Current approach uses 3 predefined waypoints for each leg's trajectory. The waypoints 
are defined as follows:
\begin{itemize}
    \item Start Point: (0, 0, 0) - The initial position of the foot.
    \item Mid Point: (Step Lenght/2, 0, Step Height) - The highest point of the swing phase.
    \item End Point: (Step Lenght, 0, 0) - The final position of the foot.
\end{itemize}

Current version uses these settings for the step parameters:
\begin{itemize}
    \item Step Lenght = 0.22
    \item Step Height = 0.12
    \item Cycle Time = 0.6      --- time of full gait cycle
    \item Duty = 0.6     --- portion of the cycle when the foot is on the ground
\end{itemize}

These settings can be tweeked to achieve different walking styles and speeds. 
For example increasing the step height could help the robot to walk on uneven terrain.

\subsection{Different trajectory}

One example to tune the trajectory was to use 4 waypoints to flatten the top of the
spline curve. This was done to reduce the acceleration at the top of the swing phase,
which could lead to a more stable walking motion. Trajectory is shown in figure
\ref{fig:fast-spline-trajectory}.

\begin{figure}[htbp]
    \centering
    \includegraphics[width=0.45\textwidth]{imgs/fast_walk_trajectory.png}
    \caption{Fast walking cubic spline trajectory for a leg swing motion.}
    \label{fig:fast-spline-trajectory}
\end{figure}

Additionaly I changed the step parameters to achieve a faster walking speed:
Video of the robot walking with this trajectory can be found at
\url{https://youtu.be/nRwTH8aU0jQ}

\subsection{Spline improvements}

For this day I focused on implementing the basic spline trajectory generation and gait phase
coordination. However, there are several potential improvements that could be made to enhance
the system. The major one would be to implement a system to dynamically adjust the step parameters
based on the robot's speed and terrain. This could involve using sensors to detect the ground
and adjust the step height and length accordingly. Additionally, implementing more advanced gait patterns,
such as walking, trotting, and galloping, could provide more versatility to the robot's movement capabilities. Future work could also explore the use of machine learning techniques to optimize the gait
patterns based on feedback from the robot's performance. 