First major problem I needed to solve is to implementation of inverse kinematics 
for the Unitree Go2 robot. Very helpful resource was an online article explaining the
inverse kinematics for this specific robot \cite{go2_inverse_kinematics}. The article 
provided the dimmensions of the robot's leg segments. Then I using geometric approach
I derived the equations to calculate the joint angles based on the desired foot position
in the robot's leg coordinate frame. 

\begin{figure}[htbp]
    \centering
    \includegraphics[width=0.45\textwidth]{imgs/go2_leg_dims.png}
    \caption{Unitree GO2 leg dimensions used for inverse kinematics calculations.}
    \label{fig:go2-leg-dim}
\end{figure}

Implementation of the inverse kinematics is in the file \texttt{GO2\_IK.py}.